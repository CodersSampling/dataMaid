\documentclass[article]{jss}

\usepackage[nogin]{Sweave}
\usepackage{pdfpages}


%%%%%%%%%%%%%%%%%%%%%%%%%%%%%%
%% declarations for jss.cls %%%%%%%%%%%%%%%%%%%%%%%%%%%%%%%%%%%%%%%%%%
%%%%%%%%%%%%%%%%%%%%%%%%%%%%%%

%% almost as usual
\author{Achim Zeileis\\Universit\"at Innsbruck \And
        Second Author\\Plus Affiliation}
\title{A Capitalized Title: Something about a Package \pkg{cleanR}}

%% for pretty printing and a nice hypersummary also set:
\Plainauthor{Achim Zeileis, Second Author} %% comma-separated
\Plaintitle{A Capitalized Title: Something about a Package cleanR} %% without formatting
\Shorttitle{\pkg{cleanR}: A Capitalized Title} %% a short title (if necessary)

%% an abstract and keywords
\Abstract{
  The abstract of the article.
}
\Keywords{keywords, comma-separated, not capitalized, \proglang{Java}}
\Plainkeywords{keywords, comma-separated, not capitalized, Java} %% without formatting
%% at least one keyword must be supplied

%% publication information
%% NOTE: Typically, this can be left commented and will be filled out by the technical editor
%% \Volume{50}
%% \Issue{9}
%% \Month{June}
%% \Year{2012}
%% \Submitdate{2012-06-04}
%% \Acceptdate{2012-06-04}

%% The address of (at least) one author should be given
%% in the following format:
\Address{
  Achim Zeileis\\
  Department of Statistics and Mathematics\\
  Faculty of Economics and Statistics\\
  Universit\"at Innsbruck\\
  6020 Innsbruck, Austria\\
  E-mail: \email{Achim.Zeileis@uibk.ac.at}\\
  URL: \url{http://eeecon.uibk.ac.at/~zeileis/}
}
%% It is also possible to add a telephone and fax number
%% before the e-mail in the following format:
%% Telephone: +43/512/507-7103
%% Fax: +43/512/507-2851

%% for those who use Sweave please include the following line (with % symbols):
%% need no \usepackage{Sweave.sty}

%% end of declarations %%%%%%%%%%%%%%%%%%%%%%%%%%%%%%%%%%%%%%%%%%%%%%%


\begin{document}

\section{Introduction}
Statisticians and data analysts spend a large portion of their time on
data cleaning and on data wrangling. Packages such as \pkg{}, \pkg{},
and \pkg{} make data wrangling a lot easier in R, but there are only a
few options available for data cleaning.


Data cleaning is a time consuming endeavour and it inherently requires
human interaction since every dataset is different and the variables
in the dataset can only be understood in the proper context of the
experiment. While each dataset is different and require
... (``simple questions''

... there are a number of common occurrences that .... This is particulary
Automated


... partly be done by ... However ... enough ... similar tasks that
pop


However it is not uncommon for ... to run a set of ....


The manuscript is structured as follows. In

%% include your article here, just as usual
%% Note that you should use the \pkg{}, \proglang{} and \code{}
%% commands.

\section{Checking a dataset for errors} \label{sec:example1}


\begin{Schunk}
\begin{Sinput}
> library(cleanR)
> data(testData)
> testData
\end{Sinput}
\begin{Soutput}
   charVar factorVar numVar intVar boolVar keyVar emptyVar _joeVar jack__var
1        a         a      1      1    TRUE      1        1       1         1
2        b         b      2      2   FALSE      2        1       2         2
3        c         c      3      3    TRUE      3        1       3         3
4        a         a      4      4    TRUE      4        1       4         4
5        b         b      5      5    TRUE      5        1       5         5
6        d         d      6      6   FALSE      6        1       6         6
7        a         a      7      7    TRUE      7        1       7         7
8        a         a      8      8   FALSE      8        1       8         8
9        b         b      9      9    TRUE      9        1       9         9
10       c         c     10     10    TRUE     10        1      10        10
11       a         a      1      1    TRUE     11        1      11        11
12       b         b      1      1   FALSE     12        1      12        12
13       d         d      1      1      NA     13        1      13        13
14       a         a      5      5      NA     14        1      14        14
15    <NA>       999      5      5      NA     15        1      15        15
   numOutlierVar smartNumVar      cprVar   cprKeyVar miscodedMissingVar
1              1           0 010101-1111 010101-1111                  .
2              2           0 020102-2929 020102-2929                   
3              3           0 121201-1902 121201-1902                nan
4              4           0 030729-2222 030729-2222                NaN
5              5           0 080909-1212 080909-1212                NAN
6              6           0 010101-1111 020202-0101                 na
7              7           0 020102-2929 030303-0101                 NA
8              8           1 121201-1902 040404-0101                 Na
9              9           1 030729-2222 050505-0101                Inf
10            10           1 080909-1212 060606-0101                inf
11            11           1 010101-1111 070707-0101               -Inf
12            12           1 020102-2929 080808-0101               -inf
13            13           1 121201-1902 090909-0101                  -
14            14           1 030729-2222 020202-0202                  9
15           100           1 080909-1212 030303-0202                  9
\end{Soutput}
\end{Schunk}


\begin{Schunk}
\begin{Sinput}
> clean(testData)
\end{Sinput}
\end{Schunk}


\includepdf[fitpaper=true, pages=-]{test.pdf}

Arguments

\section{The structure of \pkg{cleanR}} \label{sec:internals}


\section{Extending the error checks} \label{sec:extending}

\section[About Java]{About \proglang{Java}}
%% Note: If there is markup in \(sub)section, then it has to be escape as above.

\end{document}
