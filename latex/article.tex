\documentclass[article]{jss}

%%%%%%%%%%%%%%%%%%%%%%%%%%%%%%
%% declarations for jss.cls %%%%%%%%%%%%%%%%%%%%%%%%%%%%%%%%%%%%%%%%%%
%%%%%%%%%%%%%%%%%%%%%%%%%%%%%%

%% almost as usual
\author{Achim Zeileis\\Universit\"at Innsbruck \And
        Second Author\\Plus Affiliation}
\title{A Capitalized Title: Something about a Package \pkg{cleanR}}

%% for pretty printing and a nice hypersummary also set:
\Plainauthor{Achim Zeileis, Second Author} %% comma-separated
\Plaintitle{A Capitalized Title: Something about a Package cleanR} %% without formatting
\Shorttitle{\pkg{cleanR}: A Capitalized Title} %% a short title (if necessary)

%% an abstract and keywords
\Abstract{
  The abstract of the article.
}
\Keywords{keywords, comma-separated, not capitalized, \proglang{Java}}
\Plainkeywords{keywords, comma-separated, not capitalized, Java} %% without formatting
%% at least one keyword must be supplied

%% publication information
%% NOTE: Typically, this can be left commented and will be filled out by the technical editor
%% \Volume{50}
%% \Issue{9}
%% \Month{June}
%% \Year{2012}
%% \Submitdate{2012-06-04}
%% \Acceptdate{2012-06-04}

%% The address of (at least) one author should be given
%% in the following format:
\Address{
  Achim Zeileis\\
  Department of Statistics and Mathematics\\
  Faculty of Economics and Statistics\\
  Universit\"at Innsbruck\\
  6020 Innsbruck, Austria\\
  E-mail: \email{Achim.Zeileis@uibk.ac.at}\\
  URL: \url{http://eeecon.uibk.ac.at/~zeileis/}
}
%% It is also possible to add a telephone and fax number
%% before the e-mail in the following format:
%% Telephone: +43/512/507-7103
%% Fax: +43/512/507-2851

%% for those who use Sweave please include the following line (with % symbols):
%% need no \usepackage{Sweave.sty}

%% end of declarations %%%%%%%%%%%%%%%%%%%%%%%%%%%%%%%%%%%%%%%%%%%%%%%


\begin{document}

\section{Introduction}
Statisticians and data analysts spend a large portion of their time on
data cleaning and on data wrangling. Packages such as \pkg{}, \pkg{},
and \pkg{} make data wrangling a lot easier in R, but there are only a
few options available for data cleaning.


Data cleaning is a time consuming endeavour and it inherently requires
human interaction since every dataset is different and the variables
in the dataset can only be understood in the proper context of the
experiment. While each dataset is different and require
... (``simple questions''

... there are a number of common occurrences that .... This is particulary
Automated


... partly be done by ... However ... enough ... similar tasks that
pop


However it is not uncommon for ... to run a set of ....


\section{Checking a dataset for errors}

%% include your article here, just as usual
%% Note that you should use the \pkg{}, \proglang{} and \code{}
%% commands.

\begin{Schunk}
\begin{Sinput}
> library(cleanR)
> data(testData)
> testData